\documentclass[sigplan,review,dvipsnames,screen,10pt]{acmart}
\usepackage[T1]{fontenc}
\usepackage[utf8]{inputenc}
\citestyle{acmauthoryear}
\usepackage[capitalize, noabbrev]{cleveref}

%% Rights management information. SAMPLE value until references are given.
\setcopyright{acmlicensed}
\copyrightyear{2018}
\acmYear{2018}
\acmDOI{XXXXXXX.XXXXXXX}

%% These commands are for a JOURNAL article.
\acmJournal{JACM}
\acmVolume{37}
\acmNumber{4}
\acmArticle{111}
\acmMonth{8}

%%
%% end of the preamble, start of the body of the document source.
\begin{document}

%%
%% The "title" command has an optional parameter,
%% allowing the author to define a "short title" to be used in page headers.
\title{Functional (un)unparsing meets structured data}
\subtitle{Continuations for sums and products}

\author{Mathieu Boespflug}
\email{m@tweag.io}
% \orcid{1234-5678-9012}

\author{Arnaud Spiwack}
\email{arnaud.spiwack@tweag.io}
\orcid{0000-0002-5985-2086}
%% TODO: country is mandatory, not sure what we should write
%% TODO: this is the official way to make a shared affiliation, but it
%% doesn't look great at all.
\affiliation{%
  \institution{Tweag}
  % \city{??}
  \country{France}
}


%% If needed
% \renewcommand{\shortauthors}{Boespflug \& Spiwack}

\begin{abstract}
  Bidirectional parsers done more betterer.
\end{abstract}

%%
%% The code below is generated by the tool at http://dl.acm.org/ccs.cfm.
%% TODO: Please copy and paste the code instead of the example below.
%%
\begin{CCSXML}
<ccs2012>
 <concept>
  <concept_id>00000000.0000000.0000000</concept_id>
  <concept_desc>Do Not Use This Code, Generate the Correct Terms for Your Paper</concept_desc>
  <concept_significance>500</concept_significance>
 </concept>
 <concept>
  <concept_id>00000000.00000000.00000000</concept_id>
  <concept_desc>Do Not Use This Code, Generate the Correct Terms for Your Paper</concept_desc>
  <concept_significance>300</concept_significance>
 </concept>
 <concept>
  <concept_id>00000000.00000000.00000000</concept_id>
  <concept_desc>Do Not Use This Code, Generate the Correct Terms for Your Paper</concept_desc>
  <concept_significance>100</concept_significance>
 </concept>
 <concept>
  <concept_id>00000000.00000000.00000000</concept_id>
  <concept_desc>Do Not Use This Code, Generate the Correct Terms for Your Paper</concept_desc>
  <concept_significance>100</concept_significance>
 </concept>
</ccs2012>
\end{CCSXML}

\ccsdesc[500]{Do Not Use This Code~Generate the Correct Terms for Your Paper}
\ccsdesc[300]{Do Not Use This Code~Generate the Correct Terms for Your Paper}
\ccsdesc{Do Not Use This Code~Generate the Correct Terms for Your Paper}
\ccsdesc[100]{Do Not Use This Code~Generate the Correct Terms for Your Paper}

%%
%% TODO: Keywords. The author(s) should pick words that accurately describe
%% the work being presented. Separate the keywords with commas.
\keywords{Do, Not, Us, This, Code, Put, the, Correct, Terms, for,
  Your, Paper}

\maketitle

\section{Related work}

\begin{itemize}
\item Partial monadic profunctor for bidirectional parsers (and more)
  \cite{monadic-profunctors}. One lesson we can learn from this is
  that parsing is easy (in all their example one of the sides is
  easy). The difficult bit is to build a printer type which will align
  with the parser type.
\item Use partial monadic profunctor technique for random generators \cite{reflect-random}
\end{itemize}

\bibliographystyle{ACM-Reference-Format}
\bibliography{bibliography}{}

\end{document}
